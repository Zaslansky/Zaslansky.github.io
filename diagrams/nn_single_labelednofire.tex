
%!TEX program = xelatex
\documentclass{standalone}
\usepackage{tikz}
\usetikzlibrary{arrows}
\usepackage{fontspec}
    \setmainfont{Charis SIL}
\begin{document}
\def\layersep{2.5cm}

\begin{tikzpicture}[shorten >=1pt,->,draw=black!50, node distance=\layersep]
    \tikzstyle{every pin edge}=[<-,shorten <=1pt]
    \tikzstyle{neuron}=[circle,fill=black!25,minimum size=17pt,inner sep=0pt]
    \tikzstyle{input neuron}=[neuron, fill=green!50];
    \tikzstyle{output neuron}=[neuron, fill=red!50];
    \tikzstyle{hidden neuron}=[neuron, fill=blue!50];
    \tikzstyle{annot} = [text width=3em, text centered]

    % Draw the input neuron
    \node[input neuron] (I-1) at (0,-1) {};

    % Draw the single neuron in the hidden layer
    \node[hidden neuron] (H1-1) at (\layersep,-1) {};

    % Draw the output neuron
    \node[output neuron] (O-1) at (2*\layersep,-1) {};

    % Connect input neuron to the hidden neuron with label
    \path (I-1) edge node[above] {-0.75} (H1-1);

    % Connect hidden neuron to the output neuron with label
    \path (H1-1) edge node[above] {0} (O-1);

    % Annotate the layers
    \node[annot,above of=H1-1, node distance=1cm] (hl) {ReLU \\ W=1.5 \\ B=0};
    \node[annot,left of=hl] {Input};
    \node[annot,right of=hl] {Output};

\end{tikzpicture}

\end{document}
